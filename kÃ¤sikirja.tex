\documentclass{scrartcl}
\usepackage[finnish]{babel}
\usepackage{minted}
\newmintinline{c}{breaklines}

\title{Nctietue}

\begin{document}
\maketitle

\section{Yleistä}
Määritellyt tietueet ovat nimeltään \cinline/nct_var; nct_dim; nct_vset/.
Funktioitten nimet alkavat aina etuliitteellä "nct\_"
Etuliite "nct\_" esiintyy funktion nimessä vain kerran alussa.
Esimerkiksi nct\_var on alun jälkeen vain var.
Lisäksi funktioitten nimissä käytetään muuttujan tyypin tapaan nimeä \cinline/nct_coord/.
Jos funktion nimessä on nct\_dim, muuttujaa käsitellään unohtaen sillä oleva coordv-niminen osoitin nct\_var-muuttujaan.
Nct\_coord tarkoittaa nct\_dim muuttujaa, mutta funktio huomioi coordv-jäsenen toisin kuin nct\_dim-funktiot.

Muuttujan alustusfunktioista on aina olemassa kaksi versiota, joista toisen päätteenä on \_gd (given destination).
Gd-funktion ensimmäinen argumentti on viite tietueeseen, johon muuttuja sijoitetaan.
Gd-päätteetön funktio alustaa muistin luotavalle muuttujalle ja kutsuu sitten gd-funktiota.

\section{Muistinhallinta}
Funktion nimestä käy aina ilmi, jos dataa kopioidaan.
Vapauttaminen vapauttaa kaiken osoittimen viittaaman, mutta ei itse osoitinta.
Tietueissa on muuttuja \cinline/freeable_name/, jonka totuusarvo määrittää vapautetaanko tietueen nimi.

Koordinaatin alustus ja vapautus tapahtuu kasamuistia käyttäen näin:
Jos tietueeseen laitetaan pinomuistia, se täytyy poistaa tietueesta ennen vapautusfunktion kutsumista.
Koordinaatin alustus ja vapautus tapahtuu pinomuistia käyttäen näin:
\end{minted}

\end{document}

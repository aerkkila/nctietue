\documentclass{scrartcl}
\usepackage[finnish]{babel}
\usepackage{minted}
\newmintinline{c}{breaklines}

\title{Nctietue}

\begin{document}
\maketitle

\section{Yleistä}
Määritellyt tietueet ovat nimeltään \cinline/nct_var; nct_dim; nct_vset/.
Funktioitten nimet alkavat aina etuliitteellä "nct\_"
Etuliite "nct\_" esiintyy funktion nimessä vain kerran alussa.
Esimerkiksi nct\_var on alun jälkeen vain var.
Lisäksi funktioitten nimissä käytetään muuttujan tyypin tapaan nimeä \cinline/nct_coord/.
Jos funktion nimessä on nct\_dim, muuttujaa käsitellään unohtaen sillä oleva coordv-niminen osoitin nct\_var-muuttujaan.
Nct\_coord tarkoittaa nct\_dim muuttujaa, mutta funktio huomioi coordv-jäsenen toisin kuin nct\_dim-funktiot.

Muuttujan alustusfunktioista on aina olemassa kaksi versiota, joista toisen päätteenä on \_gd (given destination).
Gd-funktion ensimmäinen argumentti on viite tietueeseen, johon muuttuja sijoitetaan.
Gd-päätteetön funktio alustaa muistin luotavalle muuttujalle ja kutsuu sitten gd-funktiota.
Vain ensimmäinen taso voidaan antaa valmiina.
Esimerkiksi nct\_coord-tyypin alustaville funktioille voidaan antaa nct\_dim-muuttuja, mutta ei sen osoittamaa nct\_var-muuttujaa.
\begin{minted}{c}
  nct_dim* nct_coordcpy_gd(nct_dim* dest, const nct_dim* src) {
    nct_dimcpy_gd(dest, src);
    dest->coordv = nct_varcpy(src->coordv);
    return dest;
  }
  nct_dim* nct_coordcpy(const nct_dim* src) {
    nct_dim* dest = malloc(sizeof(nct_dim));
    return nct_coordcpy_gd(dest, src);
  }
\end{minted}

\section{Muistinhallinta}
Funktion nimestä käy aina ilmi, jos dataa kopioidaan.
Vapauttaminen vapauttaa kaiken osoittimen viittaaman, mutta ei itse osoitinta.

Koordinaatin alustus ja vapautus tapahtuu kasamuistia käyttäen näin:
\begin{minted}{c}
  int *xdata = calloc(5,sizeof(int));
  nct_dim* x = nct_to_coord(xdata, 5, NC_INT, strdup("x"));
  free_nct_coord(x);
  free(x);
\end{minted}
Jos tietueeseen laitetaan pinomuistia, se täytyy poistaa tietueesta ennen vapautusfunktion kutsumista.
Koordinaatin alustus ja vapautus tapahtuu pinomuistia käyttäen näin:
\begin{minted}{c}
  int xdata[5] = {0};
  nct_dim x = {0};
  to_nct_coord_gd(&x, nct_to_coord, 5, NC_INT, "x");
  /*poistetaan viitteet pinomuistiin ennen vapautusfunktion kutsumista*/
  x.name = NULL;
  x.coordv->name = NULL;
  x.coordv->data = NULL;
  free_nct_coord(&x);
\end{minted}

\end{document}

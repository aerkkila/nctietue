\documentclass{scrartcl}
\usepackage[finnish]{babel}
\usepackage{minted}
\newmintinline{c}{breaklines}

\title{Nctietue}

\begin{document}
\maketitle

\section{Periaatteet}
Määritellyt tietueet ovat nimeltään \cinline/nct_var; nct_dim; nct_vset/.

Jos funktio operoi vain yhden tyyppiseen muuttujaan, tyypin nimi esiintyy funkion nimessä sellaisenaan:
\begin{minted}{c}
  void free_nct_dim(nct_dim*);
  nct_vset* nct_vsetcpy(nct_vset*);
\end{minted}
Jos tyyppejä on useampi, funktion nimessä jätetään pois nct\_-etuliite:
\begin{minted}{c}
  nct_var* vararr_pluseq(nct_var*, void*);
\end{minted}

Lisäksi funktioitten nimissä käytetään muuttujan tyypin tapaan nimeä \cinline/nct_coord/.
Jos funktion nimessä on nct\_dim, muuttujaa käsitellään unohtaen sillä oleva coordv-niminen osoitin nct\_var-muuttujaan.
Nct\_coord tarkoittaa nct\_dim muuttujaa, mutta funktio huomioi coordv-jäsenen ja olettaa sen viittaavan muuttujaan (ei NULL).

\section{Muistinhallinta}
Funktion nimestä käy aina ilmi, jos dataa kopioidaan.
Poikkeus on nc\_coord-tyypin funktiot, joissa nimi kopioidaan alla alevaan muuttujaan.
Vapauttaminen vapauttaa kaiken osoittimen viittaaman, mutta ei itse osoitinta.
Alustusfunktioitten ensimmäinen argumentti on aina \cinline/dest_type* dest/.
Sen ollessa NULL alustetaan uutta muistia.
Muuten alustettava asia sijoitetaan dest-osoittimen viittaamaan muistiin.

Tässä muistinhallinta on oikein:
\begin{minted}{c}
  int *xdata = calloc(5,sizeof(int));
  nct_dim* x = to_nct_coord(NULL, xdata, 5, NC_INT, strdup("x"));
  free_nct_coord(x);
  free(x); //vapautusfunktio ei vapauta itse osoitinta
\end{minted}
Nimi vapautetaan vapautusfunktiossa, jonka vuoksi käytettin strdup-funktiota.
Jos tietueeseen laitetaan pinomuistia, se täytyy piilottaa ennen vapautusfunktion kutsumista.
Tässäkin muistinhallinta on oikein:
\begin{minted}{c}
  int xdata[] = {1,4,9,16,25};
  nct_dim x = {0};
  to_nct_coord(&x, xdata, 5, NC_INT, "x");
  x.name = NULL; //muuten tämä yritettäisiin vapauttaa free()-funktiolla
  x.coordv->data = NULL; //muuten tämä yritettäisiin vapauttaa free()-funktiolla
  free_nct_coord(&x);
\end{minted}

\end{document}
